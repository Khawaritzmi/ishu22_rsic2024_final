\documentclass[journal,article,submit,pdftex,moreauthors]{Definitions/mdpi} 

\Title{EcoActivity: Aplikasi Rekomendasi Aktivitas Berbasis Cuaca}

% MDPI internal command: Title for citation in the left column
\TitleCitation{EcoActivity: Aplikasi Rekomendasi Aktivitas Berbasis Cuaca}

% Authors, for the paper (add full first names)
\Author{Kevin Al Gazali $^{1,\dagger,\ddagger}$, Shaff Shalihin $^{2,\ddagger}$, Vicky Jesflinto $^{2}$, Ahmad Fauzhan Ramadhan $^{3}$, Zefanya Farrel Palinggi $^{3}$, dan Muh. Shofwan Siswandi $^{3,}$*}

%\longauthorlist{yes}

% MDPI internal command: Authors, for metadata in PDF
\AuthorNames{Kevin Al Gazali, Shaff Shalihin, Vicky Jesflinto, Ahmad Fauzhan Ramadhan, Zefanya Farrel Palinggi, Muh. Shofwan Siswandi}

% MDPI internal command: Authors, for citation in the left column
\AuthorCitation{Gazali, K. A.; Shalihin, S.; Jesflinto, V.; Ramadhan, A. F.; Palinggi, Z. F.; Siswandi, M. S.}
% If this is a Chicago style journal: Lastname, Firstname, Firstname Lastname, and Firstname Lastname.

% Affiliations / Addresses (Add [1] after \address if there is only one affiliation.)
\address{%
$^{1}$ \quad Universitas Hasanuddin; gazalika22h@student.unhas.ac.id \\
$^{2}$ \quad Universitas Hasanuddin;shalihins22h@student.unhas.ac.id\\
$^{3}$ \quad Universitas Hasanuddin;jesflintov22h@student.unhas.ac.id\\
$^{4}$ \quad Universitas Hasanuddin;bafr22h@student.unhas.ac.id\\
$^{5}$ \quad Universitas Hasanuddin;palinggizf22h@student.unhas.ac.id\\
$^{6}$ \quad Universitas Hasanuddin;siswandims20h@student.unhas.ac.id\\
}
% Contact information of the corresponding author
\corres{Correspondence: gazalika22h@student.unhas.ac.id;Tel.: +62 822-6064-0071}

% Current address and/or shared authorship
\firstnote{Current address:  Program Studi Sistem Informasi, Universitas Hasanuddin, Makassar, Indonesia.}  % Current address should not be the same as any items in the Affiliation section.
% The commands \thirdnote{} till \eighthnote{} are available for further notes

%\simplesumm{} % Simple summary

%\conference{} % An extended version of a conference paper

% Abstract (Do not insert blank lines, i.e. \\) 
\abstract{Cuaca memengaruhi keputusan aktivitas manusia sehari-hari, dan dengan perkembangan teknologi, aplikasi berbasis cuaca dapat memberikan rekomendasi yang lebih akurat. Tujuan dari penelitian ini adalah untuk mengembangkan aplikasi *EcoActivity*, yang memberikan rekomendasi aktivitas berdasarkan kondisi cuaca terkini. Aplikasi ini menggunakan data cuaca real-time yang diperoleh melalui API cuaca dan diproses dengan algoritma pembelajaran mesin untuk memberikan prediksi cuaca yang akurat. Sistem ini juga menyediakan rekomendasi aktivitas yang disesuaikan dengan kondisi cuaca saat itu. Hasil pengujian menunjukkan bahwa aplikasi mampu memberikan rekomendasi aktivitas yang relevan berdasarkan cuaca dan menghasilkan prediksi cuaca yang cukup akurat dalam jangka pendek. Pengujian aplikasi juga mengungkapkan antarmuka pengguna yang mudah digunakan dan responsif. Aplikasi *EcoActivity* menunjukkan potensi yang baik dalam membantu pengguna memilih aktivitas yang sesuai dengan cuaca saat itu dan memberikan prediksi cuaca yang berguna. Aplikasi ini dapat dikembangkan lebih lanjut dengan fitur-fitur tambahan, seperti integrasi dengan data cuaca lebih panjang atau analisis prediksi berbasis lokasi.
}

% Keywords
\keyword{Cuaca; Aplikasi; Prediksi Cuaca; Pembelajaran Mesin; Rekomendasi Aktivitas; Sistem Cerdas; Data Real-Time; Teknologi Cuaca; Sistem Informasi} 


%%%%%%%%%%%%%%%%%%%%%%%%%%%%%%%%%%%%%%%%%%
\begin{document}
\section{Project Charter}

\subsection{Project Title}
*EcoActivity: Aplikasi Rekomendasi Aktivitas Berbasis Cuaca*

\subsection{Project Purpose and Objectives}

Proyek *EcoActivity* bertujuan untuk mengembangkan sebuah aplikasi pintar yang memberikan rekomendasi aktivitas berdasarkan kondisi cuaca terkini serta prediksi cuaca mendatang. Aplikasi ini memungkinkan pengguna untuk merencanakan kegiatan sehari-hari dengan lebih baik, dengan memanfaatkan data cuaca real-time dan algoritma rekomendasi yang adaptif. Tujuan utama proyek ini adalah:
\begin{itemize}
    \item Menyediakan rekomendasi aktivitas harian yang sesuai dengan kondisi cuaca.
    \item Mengintegrasikan data cuaca real-time dari API eksternal untuk memberikan informasi yang akurat.
    \item Mengembangkan sistem prediksi cuaca berbasis data historis untuk mendukung perencanaan aktivitas jangka pendek.
\end{itemize}

\subsection{Scope of the Project}

Ruang lingkup proyek *EcoActivity* mencakup:
\begin{itemize}
    \item Pengembangan aplikasi berbasis Android Studio sebagai antarmuka pengguna.
    \item Integrasi dengan API cuaca eksternal (Tomorrow API) untuk data cuaca terkini.
    \item Implementasi model prediksi cuaca sederhana menggunakan dataset cuaca historis.
    \item Penyediaan fitur rekomendasi aktivitas berbasis cuaca yang memanfaatkan aturan tertentu (rule-based).
\end{itemize}

\subsection{Project Team}

Tim proyek *EcoActivity* terdiri dari enam anggota dengan peran berikut:
\begin{itemize}
    \item Kevin Al Gazali : UI/UX Designer.
    \item Shaff Shalihin : Backend Developer.
    \item Zefanya Farrel : Backend Developer.
    \item Muh.Shofwan Siswandi : Backend Developer.
    \item Vicky Jesflinto : Frontend Developer.
    \item Ahmad Fauzan Ramadhan : Frontend Developer.
\end{itemize}
\subsection{Project Milestones and Timeline}

Berikut adalah tonggak penting dan garis waktu yang direncanakan untuk proyek *EcoActivity*:
\begin{itemize}
    \item Perencanaan - Minggu 1-4
    \item Pengumpulan Data - Minggu 5-6
    \item Pengembangan - Minggu 7-10
    \item Pengujian - Minggu 11-14
    \item Deployment - Minggu 15
    \item Pengujian dan Validasi Sistem - Minggu 13-14
    \item Penyusunan Laporan - Minggu 16
\end{itemize}
\subsection{Assumptions and Constraints}

Proyek *EcoActivity* bergantung pada asumsi dan batasan berikut:
\begin{itemize}
    \item Ketersediaan akses API cuaca eksternal untuk data real-time.
    \item Akurasi prediksi cuaca terbatas oleh jumlah dan kualitas dataset historis yang digunakan.
    \item Aplikasi harus user-friendly untuk pengguna tanpa latar belakang teknis.
\end{itemize}

\subsection{Risks and Mitigations}

Beberapa risiko yang diidentifikasi beserta langkah mitigasinya adalah sebagai berikut:
\begin{itemize}
    \item Keterbatasan Data Cuaca : Jika data sulit diakses, API alternatif akan dicari.
    \itemKetidakakuratan Prediksi Cuaca : Model akan diuji dan disesuaikan sesuai hasil evaluasi.
    \item Masalah Integrasi API : Prosedur debugging akan dilakukan dan dokumentasi API dikaji.
\end{itemize}
\section{Introduction}

Cuaca memainkan peran yang sangat penting dalam kehidupan sehari-hari manusia. Aktivitas luar ruangan, perencanaan perjalanan, dan bahkan keputusan terkait kesehatan sering kali bergantung pada kondisi cuaca saat itu. Namun, kebanyakan aplikasi cuaca yang tersedia saat ini hanya memberikan informasi dasar tentang cuaca tanpa menawarkan rekomendasi aktivitas yang relevan berdasarkan kondisi cuaca yang terjadi. Oleh karena itu, penelitian ini bertujuan untuk mengembangkan aplikasi *EcoActivity*, yang memberikan rekomendasi aktivitas berbasis cuaca terkini dan juga menyediakan prediksi cuaca untuk jangka pendek.

Tujuan dari penelitian ini adalah untuk menciptakan sebuah aplikasi yang dapat memberikan rekomendasi aktivitas secara real-time yang disesuaikan dengan kondisi cuaca saat ini. Aplikasi ini memanfaatkan data cuaca real-time yang diperoleh melalui API cuaca dan menggunakan model pembelajaran mesin untuk memprediksi kondisi cuaca di masa depan. Dengan integrasi data cuaca dan analisis prediksi, aplikasi ini bertujuan untuk memberikan pengalaman yang lebih baik bagi pengguna dalam memilih aktivitas yang sesuai dengan cuaca, sehingga mereka dapat membuat keputusan yang lebih tepat dalam kegiatan sehari-hari mereka.

Secara keseluruhan, aplikasi *EcoActivity* diharapkan dapat membantu pengguna dalam mengambil keputusan terkait aktivitas harian mereka dengan lebih baik berdasarkan kondisi cuaca yang ada. Fokus utama dari penelitian ini adalah pada pengembangan dan pengujian prototipe aplikasi, serta evaluasi efektivitas aplikasi dalam memberikan rekomendasi yang bermanfaat dan relevan..
%%%%%%%%%%%%%%%%%%%%%%%%%%%%%%%%%%%%%%%%%%
\section{Materials and Methods}

Bagian ini menjelaskan metode yang digunakan dalam pengembangan dan evaluasi aplikasi *EcoActivity*. Proyek ini bertujuan untuk memberikan rekomendasi aktivitas berdasarkan cuaca terkini dan juga menyediakan fitur prediksi cuaca untuk membantu pengguna memilih aktivitas yang sesuai dengan kondisi cuaca saat ini.

\subsection{Data Collection and Explanation}
Data yang digunakan dalam aplikasi EcoActivity diperoleh dari tiga sumber utama yaitu :
\begin{itemize}
    \item Tomorrow API : API ini menyediakan data cuaca real-time yang mencakup parameter seperti suhu, kelembapan, kecepatan angin, dan kondisi cuaca (misalnya cerah, hujan, berawan), yang digunakan untuk memperbarui kondisi cuaca dalam aplikasi
    \item Dataset weather type classification dari kaggle: menyediakan data historis cuaca yang digunakan untuk melatih model prediksi cuaca dan analisis data.
    \item Data aktivitas responden :  berisi informasi tentang preferensi dan kebiasaan aktivitas fisik pengguna, yang digunakan untuk memberikan rekomendasi aktivitas berdasarkan kondisi cuaca saat ini.
\end{itemize}
Ketiga sumber data ini bekerja bersama-sama untuk menghasilkan rekomendasi aktivitas yang lebih relevan dan sesuai dengan cuaca saat ini.

\subsection{Algorithm or Model}
Aplikasi *EcoActivity* menggunakan dua komponen utama dalam memberikan rekomendasi aktivitas dan prediksi cuaca: integrasi dengan API cuaca untuk mendapatkan informasi terkini dan penggunaan model prediksi berbasis dataset cuaca historis.

1. Rekomendasi Aktivitas Berdasarkan API Cuaca : 
   Aplikasi terhubung dengan Tomorrow API untuk mendapatkan data cuaca real-time, termasuk suhu, kelembapan, kecepatan angin, dan kondisi cuaca (misalnya, cerah, hujan, berawan). Berdasarkan informasi cuaca yang diterima, sistem memberikan rekomendasi aktivitas yang sesuai. Sebagai contoh, jika cuaca cerah, aplikasi dapat merekomendasikan aktivitas luar ruangan seperti jogging atau bersepeda, sementara jika cuaca hujan, aplikasi dapat merekomendasikan aktivitas dalam ruangan seperti menonton film atau membaca buku.

2. Prediksi Cuaca Menggunakan Dataset Cuaca : 
   Selain mendapatkan data cuaca terkini, aplikasi *EcoActivity* juga dilengkapi dengan fitur prediksi cuaca. Untuk itu, aplikasi menggunakan model pembelajaran mesin yang dilatih dengan dataset cuaca historis. Dataset ini berisi data cuaca dari berbagai lokasi selama periode waktu tertentu. Model pembelajaran mesin, seperti Random Forest atau Support Vector Machines (SVM), digunakan untuk memprediksi kondisi cuaca di masa depan berdasarkan data yang telah dilatih. Prediksi cuaca ini memberikan perkiraan mengenai suhu, kelembapan, dan kondisi cuaca untuk beberapa jam ke depan, yang membantu aplikasi memberikan rekomendasi aktivitas berdasarkan prakiraan cuaca yang lebih akurat.

Dengan kombinasi API cuaca untuk data real-time dan model prediksi cuaca berbasis dataset historis, aplikasi *EcoActivity* mampu memberikan rekomendasi aktivitas yang tepat berdasarkan cuaca saat ini dan prediksi cuaca yang akan datang.

\subsection{Testing (Procedures and Metrics)}
Pengujian aplikasi dilakukan dalam dua tahap utama:
1. Pengujian kegunaan : melalui presentasi mingguan yang diadakan untuk memaparkan kemajuan proyek *EcoActivity* kepada seluruh anggota kelompok dan dosen pembimbing. Setiap minggu, anggota kelompok melakukan presentasi tentang fitur baru yang dikembangkan, seperti update dalam model prediksi cuaca atau penambahan fitur baru dalam aplikasi. Selama presentasi, anggota kelompok menerima umpan balik langsung dari pembimbing dan teman kelompok yang digunakan untuk meningkatkan antarmuka dan fungsi aplikasi.

Selain itu, presentasi juga berfungsi untuk mengidentifikasi masalah teknis dan memperbaiki kesalahan dalam alur penggunaan aplikasi. Evaluasi kegunaan dilakukan berdasarkan umpan balik yang diberikan selama presentasi mengenai:
- Kejelasan antarmuka pengguna
- Kemudahan dalam menggunakan aplikasi
- Efektivitas rekomendasi aktivitas yang diberikan berdasarkan cuaca
2. Pengujian Akurasi Model Prediksi Cuaca : Untuk menguji akurasi model prediksi cuaca, data hasil prediksi dibandingkan dengan data cuaca aktual yang diperoleh dari API. Metrik seperti Mean Absolute Error (MAE) dan Root Mean Square Error (RMSE) digunakan untuk mengukur seberapa baik model dapat memprediksi kondisi cuaca.

\subsection{Evaluation Metrics}
Beberapa metrik evaluasi digunakan untuk menilai kinerja aplikasi dan model prediksi cuaca:
- Akurasi Prediksi Cuaca : Diukur dengan menggunakan MAE dan RMSE untuk membandingkan prediksi cuaca dengan data cuaca aktual.
- Kepuasan Pengguna : Diukur melalui hasil survei dan wawancara yang mengevaluasi tingkat kepuasan pengguna terhadap antarmuka dan rekomendasi aktivitas.
- Kecepatan Respons Aplikasi : Mengukur waktu yang dibutuhkan aplikasi untuk menampilkan informasi cuaca dan rekomendasi aktivitas setelah permintaan pengguna.
- Relevansi Rekomendasi : Evaluasi seberapa relevan rekomendasi aktivitas yang diberikan berdasarkan kondisi cuaca yang diprediksi.

\section{Problem}

Saat ini, tidak banyak aplikasi yang menggabungkan data cuaca real-time dengan rekomendasi aktivitas yang disesuaikan untuk pengguna. Banyak aplikasi cuaca hanya menyediakan informasi cuaca tanpa menyesuaikannya dengan kebutuhan aktivitas pengguna. Masalah yang ingin dipecahkan oleh *EcoActivity* adalah bagaimana memberikan rekomendasi aktivitas yang tepat sesuai dengan kondisi cuaca terkini, sehingga pengguna dapat merencanakan kegiatan mereka dengan lebih efektif dan aman. 

Tantangan lainnya termasuk:
\begin{itemize}
    \item Menyediakan prediksi cuaca yang akurat dengan model prediksi berbasis data historis.
    \item Mengintegrasikan data cuaca real-time dari API eksternal dengan model rekomendasi yang intuitif.
    \item Memberikan antarmuka pengguna yang mudah dipahami dan mudah digunakan.
\end{itemize}

\section{Intelligence System}

*EcoActivity* memanfaatkan sistem cerdas yang mengombinasikan model prediksi cuaca dan algoritma berbasis aturan (*rule-based*) untuk menghasilkan rekomendasi aktivitas. Sistem ini terdiri dari dua komponen utama:
\begin{itemize}
    \item Model Prediksi Cuaca : Menggunakan data historis untuk memprediksi kondisi cuaca di masa mendatang.
    \item Sistem Rekomendasi Aktivitas : Algoritma berbasis aturan yang memberikan rekomendasi aktivitas berdasarkan kondisi cuaca saat ini.
\end{itemize}
Sistem cerdas ini dirancang untuk secara otomatis memperbarui rekomendasi aktivitas setiap kali ada pembaruan data cuaca, sehingga pengguna selalu mendapatkan informasi terkini.
\subsection{System Architecture}

Arsitektur sistem *EcoActivity* terdiri dari beberapa komponen inti yang bekerja secara sinergis untuk menyediakan rekomendasi aktivitas berbasis cuaca. Struktur utama sistem ini adalah sebagai berikut:

\begin{itemize}
    \item Frontend (Antarmuka Pengguna) : Dibangun menggunakan Android Studio untuk tampilan antarmuka pengguna yang interaktif. Di sini, pengguna dapat melihat informasi cuaca terkini dan mendapatkan rekomendasi aktivitas.
    \item Backend (Server dan API Integrasi): Terdapat server yang mengelola proses data cuaca dari API eksternal (Tomorrow API) dan mengirimkan data tersebut ke aplikasi.
    \item Weather Prediction Model : Model prediksi cuaca yang dilatih menggunakan data historis, yang berada di backend dan memberikan prediksi cuaca untuk beberapa jam atau hari ke depan.
    \item Rule-Based Activity Recommendation System : Sistem yang menghasilkan rekomendasi aktivitas berdasarkan kondisi cuaca saat ini.
\end{itemize}

Secara keseluruhan, arsitektur ini dirancang untuk menyediakan komunikasi real-time antara frontend dan backend serta untuk memastikan bahwa pengguna selalu memiliki akses ke informasi cuaca terkini dan prediksi yang akurat.


\subsection{System Workflow}

Pada bagian ini, dijelaskan alur kerja aplikasi EcoActivity mulai dari pengambilan data cuaca hingga rekomendasi aktivitas yang diberikan kepada pengguna. Berikut adalah tahapan-tahapan utama dalam workflow sistem ini:

\begin{enumerate}
    \item Pengambilan Data Cuaca : Saat pengguna membuka aplikasi, aplikasi mengambil data cuaca terkini dari Tomorrow API yang mencakup informasi seperti suhu, kelembapan, kecepatan angin, dan kondisi cuaca.
    \item Prediksi Cuaca : Sistem menggunakan model prediksi cuaca untuk memproyeksikan kondisi cuaca selama beberapa jam atau hari ke depan.
    \item Rekomendasi Aktivitas : Berdasarkan data cuaca saat ini, sistem rekomendasi berbasis aturan memberikan daftar aktivitas yang sesuai untuk kondisi tersebut, seperti aktivitas luar ruangan untuk hari cerah atau aktivitas dalam ruangan untuk hari hujan.
\end{enumerate}

Workflow ini memastikan bahwa aplikasi dapat memberikan rekomendasi aktivitas yang relevan dan akurat, serta memperbarui data cuaca secara berkala untuk memberikan informasi yang paling up-to-date.

%%%%%%%%%%%%%%%%%%%%%%%%%%%%%%%%%%%%%%%%%%
\section{Project Documentation}
\subsubsection{Implementation}

Bagian ini menjelaskan implementasi dari tiap komponen aplikasi *EcoActivity*.

\subsubsection{API Integration}
API Tomorrow diintegrasikan untuk mendapatkan data cuaca real-time yang diproses di backend untuk memberikan rekomendasi aktivitas.

\subsubsection{Machine Learning Model}
Model prediksi cuaca dikembangkan menggunakan Python dengan dataset yang diperoleh dari Kaggle. Model ini dilatih untuk menghasilkan prediksi cuaca jangka pendek berdasarkan data historis.

\subsubsection{User Interface (UI)}
Antarmuka pengguna dibangun di Android Studio dan dirancang untuk memberikan pengalaman yang intuitif. Tampilan utama mencakup informasi cuaca dan rekomendasi aktivitas yang relevan.

\subsubsection{Backend Functionality}
Backend berfungsi untuk mengelola data dari API serta menyediakan data prediksi cuaca kepada aplikasi. Backend juga menyimpan preferensi pengguna untuk personalisasi rekomendasi.
\section{Results and Discussion}

\subsection{Result}
\subsubsection{Hasil Training Model}
Dataset \textit{Weather Type Classification} dari Kaggle digunakan untuk melatih model prediksi cuaca pada aplikasi \textit{EcoActivity}. Hasil evaluasi menunjukkan akurasi sebesar 0.94, dengan rincian hasil evaluasi precision, recall, dan f1-score yang ditampilkan pada Tabel~\ref{tab:classification-report}. Hasil ini menunjukkan performa yang cukup baik untuk memprediksi kondisi cuaca.

\begin{table}[h]
    \centering
    \caption{Classification Report dari Model Prediksi Cuaca}
    \begin{tabular}{lcccc}
        \hline
        Kelas & Precision & Recall & F1-Score & Support \\
        \hline
        0 (Clear) & 0.82 & 0.96 & 0.88 & 622 \\
        1 (Rainy) & 0.99 & 0.94 & 0.96 & 2018 \\
        \hline
        \textbf{Accuracy} & \multicolumn{4}{c}{0.94 (Total: 2640)} \\
        Macro Avg & 0.90 & 0.95 & 0.92 & 2640 \\
        Weighted Avg & 0.95 & 0.94 & 0.94 & 2640 \\
        \hline
    \end{tabular}
    \label{tab:classification-report}
\end{table}
Model menunjukkan performa yang kuat dengan nilai f1-score tertinggi untuk kelas 1 (Rainy) sebesar 0.96, diikuti oleh kelas 0 (Clear) dengan nilai f1-score sebesar 0.88. Hasil ini mendukung integrasi model prediksi cuaca dalam aplikasi \textit{EcoActivity}, memberikan prediksi yang akurat untuk aktivitas berdasarkan kondisi cuaca.

\subsection{Discussion}
Kendala yang ditemukan selama pengujian mencakup latensi pada jaringan yang lambat dan akurasi prediksi yang dapat ditingkatkan dengan dataset yang lebih besar atau model yang lebih canggih.

\section{Conclusion}
Aplikasi *EcoActivity* berhasil mengintegrasikan data cuaca real-time dan historis dengan informasi aktivitas responden untuk memberikan rekomendasi aktivitas yang sesuai dengan kondisi cuaca terkini. Dengan memanfaatkan Tomorrow API untuk pembaruan cuaca real-time dan dataset cuaca dari Kaggle, serta informasi aktivitas responden, aplikasi ini mampu memberikan saran yang lebih relevan dan personal kepada pengguna. Selain itu, sistem prediksi cuaca yang dibangun dalam aplikasi dapat membantu meningkatkan pengalaman pengguna dengan menyediakan informasi yang lebih akurat terkait cuaca yang dapat mempengaruhi pilihan aktivitas. Kedepannya, pengembangan lebih lanjut dapat fokus pada peningkatan akurasi prediksi dan penambahan fitur yang lebih personal, seperti rekomendasi aktivitas berdasarkan kebiasaan atau lokasi pengguna, untuk meningkatkan kegunaan dan efisiensi aplikasi.



\end{document}

