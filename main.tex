\documentclass[a4paper,12pt]{report}
\usepackage{graphicx}
\usepackage{geometry}
\geometry{margin=1in}
\usepackage{setspace}
\usepackage{titlesec}

% Pengaturan heading
\titleformat{\chapter}[display]
  {\normalfont\huge\bfseries}{\chaptertitlename\ \thechapter}{20pt}{\Huge}
\titlespacing*{\chapter}{0pt}{-30pt}{20pt}

% Cover
\begin{document}
\begin{titlepage}
    \centering
    \vspace*{1cm}
    
    \includegraphics[width=0.2\textwidth]{unhas_logo.png}\\ % Logo Unhas
    \vspace{1cm}
    
    {\Large \textbf{Laporan Project Akhir}}\\
    \vspace{0.5cm}
    {\Large \textbf{Sistem Informasi Cerdas}}\\
    
    \vfill
    Disusun oleh:
    
    \begin{tabbing}
        Nama Anggota 1 \hspace{1cm} \= NIM 1 \\
        Nama Anggota 2 \> NIM 2 \\
        Nama Anggota 3 \> NIM 3 \\
        % Tambahkan anggota sesuai kebutuhan
    \end{tabbing}
    
    \vfill
    
    \textbf{Universitas Hasanuddin}\\
    \textbf{Makassar, Indonesia}\\
    \textbf{\today}
    
\end{titlepage}

\tableofcontents
\newpage

% Chapter 1: Project Charter
\chapter{Project Charter}
\section{Background}
% Tulis latar belakang proyek

\section{Objectives}
% Tulis tujuan proyek

\section{Scope}
% Tulis ruang lingkup proyek

% Chapter 2: Methods
\chapter{Methods}
\section{Data Collection}
% Jelaskan cara pengumpulan data dan sumber data

\section{Data Explanation}
% Jelaskan atribut atau fitur pada data yang digunakan

\section{Algorithm or Model}
% Deskripsi algoritma atau model yang digunakan

\section{Testing}
% Penjelasan tentang metode dan strategi pengujian yang digunakan
\subsection{Testing Procedures}
% Langkah-langkah pengujian
\subsection{Testing Metrics}
% Metrik pengujian yang digunakan untuk mengukur performa

\section{Evaluation Metrics}
% Metrik evaluasi yang digunakan untuk mengukur keberhasilan model

% Chapter 3: Problem
\chapter{Problem}
% Deskripsi masalah yang ingin diselesaikan dengan sistem

% Chapter 4: Intelligence System
\chapter{Intelligence System}
\section{System Architecture}
% Deskripsi arsitektur sistem

\section{System Workflow}
% Alur kerja atau proses dalam sistem

% Chapter 5: Project Documentation
\chapter{Project Documentation}
\section{Implementation}
% Deskripsi implementasi teknis

\section{Results and Discussion}
% Hasil dan pembahasan

\section{Conclusion}
% Kesimpulan dari proyek

\newpage
\appendix
\chapter{Appendix}
% Lampiran jika ada

\end{document}
